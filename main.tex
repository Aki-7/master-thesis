\documentclass[master]{cimt}

\usepackage{graphicx}
\usepackage{url}

\usepackage{layout}

\usepackage[dvipdfmx]{hyperref}
\usepackage{pxjahyper}

\usepackage[labelsep=period]{subcaption}
\captionsetup{labelsep=period}

\usepackage{listings,jlisting} 
\lstset{
  basicstyle={\ttfamily},
  identifierstyle={\small},
  commentstyle={\smallitshape},
  keywordstyle={\small\bfseries},
  ndkeywordstyle={\small},
  stringstyle={\small\ttfamily},
  frame={tb},
  breaklines=true,
  columns=[l]{fullflexible},
  numbers=left,
  xrightmargin=0zw,
  xleftmargin=3zw,
  numberstyle={\scriptsize},
  stepnumber=1,
  numbersep=1zw,
  lineskip=-0.5ex
}

\hypersetup{%
  colorlinks=true,
  linkcolor=black,
  citecolor=black,
  urlcolor=black
}

\jtitle{マルチコンテキスト没入環境のための\\Windowing Systemの実現に関する研究}

\etitle{The design and implementation of a windowing system\\for multi-context immersive environments.}

\studentid{48-216413}

\jauthor{木内 陽大}

\eauthor{Akihiro Kiuchi}

\supervisor{江崎 浩 教授}

\handin{2023}{1}

\begin{document}

\maketitle

\frontmatter

\begin{jabstract}
  \input src/abst-ja.tex
\end{jabstract}

\begin{eabstract}
  \input src/abst-en.tex
\end{eabstract}

\tableofcontents

\mainmatter

\input src/intro.tex
\input src/overview.tex
\input src/rendering.tex
\input src/2d.tex
\input src/conclusion.tex

\backmatter

% \pubUseLongName % 指定すると,タイトルが 「発表文献と研究活動」になる.
\begin{publications}
  \item Akihiro Kiuchi, Taishi Eguchi, Jun Rekimoto, Manabu Tsukada, and Hiroshi Esaki.
  Zigen: A windowing system enabling multitasking among 3d and 2d applications in immersive environments.
  In ACM SIGGRAPH 2022 Posters, SIGGRAPH ’22, New York, NY, USA, 2022. Association for Computing Machinery.
\end{publications}

\bibliographystyle{junsrt}
\bibliography{main}

\begin{acknowledgements}
  \input src/acknowledgements.tex
\end{acknowledgements}

% \appendix

% If needed.

\end{document}
