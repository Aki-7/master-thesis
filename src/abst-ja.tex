現代の2次元デスクトップ環境ではWindowing Systemによって,
1つのスクリーン空間に複数のアプリケーションを立ち上げて使うことができる.
ユーザがどのような目的を持ってどのような作業をしようとしているかという
ユーザのコンテキストはユーザごと,そしてユーザの時々によって変わる多様で不特定なものであるが,
ユーザはアプリケーションをインストールし,恣意的に組み合わせて用いることで
多様なコンテキストにあったスクリーン空間を作り上げることができる.

一方現状の没入環境では基本的に1つのアプリケーションがユーザの視野全体を支配している.
そのため没入環境の3D空間は,1つのアプリケーションによって会議やイベントといった特定のコンテキスト
のために設計された,シングルコンテキストな空間となっている.
本研究の目的はコンテキストが連続的に移り変わってゆく普段の仕事や生活を支えることができる,
マルチコンテキストな没入環境の実現を目指し,その新しいパラダイムに必要な基盤システムを
詳細に検討することである.
本研究を進めるにあったては実際に利用可能なXR向けのWindowing Systemである``Zwin''の
開発を同時に行い,OSSプロジェクトとしての社会実装も進めてきた.
本論文は``Zwin''実現に向けた基礎研究のうち,全体の設計方針とレンダリングプロトコル,
そして既存の2Dアプリケーションの没入環境での利用に関しての研究成果を報告する.
全体の設計方針では,類似研究の長所・短所をまとめ直し,
マルチコンテキストな没入環境に必要なシステムの設計を導いた.
レンダリングに関してはレンダリングの自由度・性能が高く,かつ質の悪いアプリケーションに対して
堅牢なレンダリングシステムを提案し,実用に十分であることを示した.
既存の2Dアプリケーションの利用に関しては,3Dアプリケーションの操作に関するプロトコルを
2D Windowing Systemのものと変換可能に設計することで,
既存の2Dアプリケーションと3Dアプリケーションとの間で
ドラッグ \& ドロップのような協調が可能であることを示した.

本研究が目指すマルチコンテキストな没入環境では,様々なベンダーのアプリケーションが同時に
起動して没入環境を作り上げてゆくという性質,また,それぞれのアプリケーションは空間全体ではなく,
特定の機能だけを提供すればよいという性質から,アプリケーション自体の多様性や,
その開発元の多様性は増してゆくと考える.本研究が発展し,没入環境がより参画しやすい,
開かれた市場となることを期待する.
