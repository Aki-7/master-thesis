\chapter{結言}

% future work
% input
% AR
%% multi-locale / locale-sync

最後に本研究がマルチコンテキストな没入環境の実現にどのように貢献したかを考察し,まとめとする.

\section{レンダリングプロトコル}

本システムで提案したレンダリングプロトコルは実験で示したとおり,実用に十分な性能を発揮した.
レンダリングの自由度に関しては,アプリケーション間でレンダリングに影響を及ぼし合うことや
マルチパスを用いたレンダリングは実現できなかったが,
リッチな見た目を要求しない,仕事や生活をアシストするシンプルなアプリケーションを
作るには十分であり,複数アプリケーションによるマルチコンテキストな没入環境を実現する
第一歩としては十分に現実的なレンダリングの自由度を得たと考える.
ただし,環境の落ち着きや,安心感を表現したりすることは重要であり,よりリッチなアプリケーションを
実現可能にするためにも,レンダリングプロトコルは今後も改善の余地を研究する必要がある.

\section{2D アプリケーションの利用}

既存手法における2Dアプリケーションの利用では,
画面共有の仕組みの上に作られていたことや,
2Dアプリケーションと3Dアプリケーションとが同時に存在する場合を考慮していなかったために,
通常のディスプレイで2Dアプリケーションを利用する場合と比べて制約があった.
マルチコンテキストな没入環境という新しいパラダイムを実現していくうえで難しい点は,
そのパラダイム上で動く3Dアプリケーションが最初に少なく,ユーザを惹きつけないということである.
2Dアプリケーションをうまく使えるようにすることは,特にこの最初期において重要であり,
本研究成果はマルチコンテキストな没入環境の実現に大きく寄与したと考える.

\section{今後の展望・展開}

本研究ではマルチコンテキストな没入環境の実現に必要なWindowing Systemにおける
レンダリングプロトコルと2Dアプリケーションの利用に関して,提案と評価をした.
しかし,マルチコンテキストな没入環境のためのWindowing Systemには
まだ多くの研究の余地がある.
例えば,複数のユーザがコラボレーションを図る場合は,どのようにアプリケーションの状態や位置を
共有するのか,どこまでWindowing Systemとしてサポートするべきかといった課題がある.
また複合現実を考えた場合は,複数のアプリケーションが同じ現実の物体に対して何らかの
レンダリングを行おうとして競合が発生したり,複数のアプリケーションがそもそも
現実のオブジェクトをどのように理解するべきか,といった課題がある.
本研究がこれらの課題を洗い出し,解決していくうえでのベースラインとなることを期待する.

また,本研究が目指すマルチコンテキストな没入環境では,
様々なベンダーのアプリケーションが同時に起動して没入環境を作り上げてゆくという性質,
また,それぞれのアプリケーションは空間全体ではなく,特定の機能だけを提供すればよいという性質から,
アプリケーション自体の多様性や,その開発元の多様性は増してゆくと考える.
本研究が発展し,没入環境がより参画しやすい,開かれた市場となることを期待する.
