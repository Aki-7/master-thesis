\chapter{緒言}

\section{背景}

% エンターテイメントを超えたXRの活用もある
Virtual Reality (VR) や Augmented Reality (AR) などの没入環境の社会実装は昨今急速に
進んでおり,比較的安価なコンシューマ向けのHead Mounted Display (HMD) の登場により,
多くの人が没入環境を体験,利用できるようになった.
没入空間の利用シーンは年々多様化しており,コンシューマ向けのVR/ARマーケットではゲームなど
エンターテイメント向けの利用が目立つが,VRChat\footnote{https://hello.vrchat.com/}や
Horizon World\footnote{https://www.oculus.com/horizon-worlds/}といった
コミュニティに重きを置いた利用も活発になり,コンピュータネットワーク上の新たな世界を指す
メタバースという言葉が流行っている.職業訓練での活用も活発であり,多くの研究や
産業界での実際の導入も進んでいる.

% 引用する。ref: https://www.sciencedirect.com/science/article/pii/S0747563221004489?casa_token=jobxsigkGQgAAAAA:ZlsQ9qSco-m6yyW-6DJVsP1eifqtqyISE8K_NyA9uzLyYr51nbUa7bbQxzWO0Bh1V6kUEftt5yhh
% 言いたいことをまとめる。 多様化が進んでいて、色々なコンテキストでの利用がある。

\section{問題点}

% ユーザの視野全体を占めている。
% シングルコンテキスト

\section{本研究の目的}

% マルチコンテキスト
% より自由な市場の実現
% 実際に動くSystemを実装している。そのデザインを詳細に述べることで、今後のベースラインになることを期待する。
