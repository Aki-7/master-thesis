\chapter{提案するシステムの概要}

\section{類似研究}

% Windowing Systemを使ったり、自分なりのIPCをしたり、Shared Libだったり。
% 古い研究が多いが、これらの特徴をまとめる。

\section{背景技術:Windowing System}

% プロセスが分かれていてOSの安全性を使えるなど。


% これらは overview の Windowing Systemのとこで使うのが良さそう。
% 2次元のデスクトップ環境ではブラウザやメールクライアントなど複数のアプリケーションを同時に
% 利用可能であるが,アプリケーション間の衝突なくこれを実現するための工夫がある.
% まず,2次元のデスクトップ環境では1つのスクリーン空間上に複数のアプリケーションが
% ウィンドウと呼ばれる長方形領域の形で置かれる.
% ウィンドウの位置や大きさは基本的にユーザが調整できるため,アプリケーションの空間的な衝突を
% ユーザ自身が最小限にできる.
% また,ユーザはマウスなどのポインティングデバイスでスクリーン上のカーソルを操作し,
% カーソルを介してアプリケーションとのインタラクションを行う.
% 入力に関するこのプロトコルはカーソルが重なっているウィンドウのみにカーソルのイベントが伝達される
% という制約ゆえに,1つのポインティングデバイスの入力イベントを適切にアプリケーションに
% 割り振っており,ユーザの意図しないアプリケーションが入力を受け取ったり,複数のアプリケーションが
% 入力を受け取ってしまったりといった,入力の衝突を最小限にしている.


\section{本論文での結論}

% Windowing Systemを利用すること。
% マルチコンピュータでの相互互助などはそれぞれのClient Applicationに任せる方針を取る。
% それ以外にもすること、しないことをはっきりさせたい。
